% -*- root: ../medrob_lab.tex -*-

% -----
% INTRODUCTION
% -----
\section{Introduction}
\hrule
\vspace{0.5cm}

\begin{tcolorbox}[width=\linewidth,frame empty]
    \begin{minipage}{\textwidth}
        \begin{center}
        {\large \textit{``I hear and I forget. I see and I remember. I do and I understand.''}}
        \end{center}

        \raggedleft
        - Confucius
    \end{minipage}
\end{tcolorbox}

The aim of the lab component of the Medical Robotics course is to give you hands on experience in building, characterizing, and controlling robots. The labs are all designed to help you apply some of the concepts you learn in the lectures in the real world. You will get to lean how to use the tools and equipment in the lab, and how to use them to build and control robots. If you do all the lab experiments well, you will gain a good understanding of the concepts taught in the course, and have gained some epxerience in designing, builing, and controlling robots for solving real world problems.

There are 6 labs in the course. The time duration for each lab experiment is variable as some of them only require you to collect and analyze data using equipment available in the departmebnt, while other require you to design, fabricate, assemble and experiment. The deadline for each lab will be announced in the course website. You will be expected to complete the lab experiments within the deadline.

You will be working in groups of 3 to 4 students. This lab manual contains all the necessary details for carrying out each lab experiment. The lab manual will contain the instructions for the lab, and the questions you need to answer. 

You will be expected to answer the questions in the lab manual and submit your lab report on or before the set deadline. The lab report should contain the answers to the questions in the lab manual, and the results of the experiments you perform in the lab. The lab report should be submitted as a PDF file on Microsoft teams.
\newpage

